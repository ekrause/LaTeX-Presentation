% this file contains examples of how to create and use new functions, with and without input arguments

\documentclass{article}
\usepackage{hyperref}

% arguments in latex functions in large part are nothing more than text strings, which are then expanded inside the function

% a function with no arguments
\newcommand{\sauerheader}[0] % the [0] indicates this function takes no arguments
{
	\begin{center}
	Eric Krause \hfill 
	\texttt{\href{https://github.com/ekrause}{<github>}} \hfill 
	\texttt{\href{http://www.sauerkrause.org}{<blag>}}
	\vspace*{-8pt}\\ 
	\hspace*{-18pt} 
	\hrulefill\\
	\end{center}
}

\newcommand{\quickequation}[2] % this function takes 2 arguments
{
	\textit{#1} $#2$ % #1 and #2 indicate where the arguments should be expanded. 
}

\begin{document}

\sauerheader % even though there are no args, empty braces are required

Here's a little function that prints its first argument in italic, then interprets its second arg in math mode.\\ 

% here we use the equation, passing each argument inside curly braces
\quickequation{Area of a Circle}{A = \pi r ^2}\\
  

% one or more arguments can be left blank, but empty curly braces for each are still required
\quickequation{}{\pi r ^2}

\end{document}